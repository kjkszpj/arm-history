%more code?
\documentclass{article}
\usepackage{hyperref}
%\usepackage{xeCJK}
\usepackage{listings}
\usepackage{multirow}
\usepackage{color}
\lstset{
	basicstyle = \small\ttfamily,
	breaklines = true,
	frame = single,
	language = [ANSI]C,
	tabsize = 2,
	showspaces = false,
	showstringspaces = false,
	breakindent =1.1em,
}
\lstset{
	basicstyle = \small\ttfamily,
	breaklines = true,
	frame = ss,
	language = bash,
	tabsize = 2,
	showspaces = false,
	showstringspaces = false,
	breakindent =1.1em,
}
%\setCJKmainfont[BoldFont={Microsoft YaHei}, ItalicFont={KaiTi}]{SimSun}
%\setCJKsansfont{BoldFont=Microsoft YaHei}
%\setCJKmonofont{KaiTi}

\title{LAKSMI\\Some OS Implementation on ARM}
\author{Peijie You 13307130325}

\begin{document}
\maketitle
\tableofcontents
\newpage

\section{Opening}

In this project for Operating System course, we are told to \emph{implement our own operating system on ARM}. However, we have not finish all features of it by now. In this report, we are going to take a review of what we have done during the semester, and then describe our version of AIM operating system (I called it LAKSMI) here.

\section{Previously on OS}

During the course, we are guided by our TA to build program we need for our final OS. We will review these thing quickly here.

\subsection{STEP 0 --- Some Introduction on ARM or the Zedboard}

Before that, I have nearly \textbf{no} experience on ARM or other architecture.

Notable things on ARM and the zedboard.
\begin{itemize}
	\item	Core register, r0-r12, sp, lr, pc, some of them are \textbf{banked}.
	\item CPSR, and SPSR, program status register, it will remember all kind of flag, and the mode.
	\item	CP15(co-processor?) thing, like SCTLR, TTBR0 .... And the special instruction MCR MRC to load/storage the register value. \emph{They are bank for the processor}.
	\item	Mode, USR, IRQ, FIQ, Supervisor, Monitor, Abort, Hyp, Undefined, System, we mainly need SYS and USR here.
	\item	Routine while exception arose, we should talk about it latter.
	\item	Complicated number is not supported, like 0x12345678, you should separate them into two operation instruction if needed. (Since the ARM instruction is \textbf{fixed size})
\end{itemize}

\subsection{STEP 1 --- run the program}

To run program on the board, you may want to \underline{compile and link} the program on your own computer and load to some specific location. (No compiler on the board).

The procedure are as follows:

\begin{enumerate}
	\item	compile with arm-gcc
	\item	link with arm-ld
	\item	load to somewhere on the SD card by ``dd'' instruction on linux.
\end{enumerate}

When the firmware program finish its work, it will load the first block in SD card to the (physical) address 0x100000. So our program should satisfy the following requirement:

\begin{itemize}
	\item	saved on the first block on sd card (can be done by proper dd instruction)
	\item	know that it will execute at address 0x100000 (done by ldscript)
	\item have length less than the MBR size(like 446B)
\end{itemize}

After this experiment, I come to know about the MBR thing.

\subsection{STEP 2 --- build bootloader}

Since the length is MBR is limited and our kernel may not fit in it, we have to \underline{load our own kernel into memory}. Here in our OS, a kernel is a ELF file stored from the first block of the \textbf{second} partition.

ELF is a file format, for Executable and Linkable Format, according to the manual, we have to:
\begin{enumerate}
	\item	read the elf header, know some information about the program header
	\item	load program header entry(a segment?), it will specified where to load this segment thing, just load as it says, I will load to the \emph{paddr} for kernel and \emph{vaddr} for user program
	\item	jump to the program entry specified in the elf header
\end{enumerate}

\emph{Noted that the validation of elf header is not implemented here.}

\subsection{STEP 3 --- memory management}

An important things that we have not brought up yet is the \textbf{MMU}, so no virtual address - physical address translation yet. Our job here is to enable MMU and clean things that we no longer need.

In ARM32, we have 2 level of page table structure, and 4 kinds of page(4KB, 64KB, 1MB, 16MB), in this project we will only use page of 4KB(for user) and 1MB(for kernel).

We divide the virtual space into \emph{2G:2G}, the lower 2G is user space is higher 2G is kernel space.

\textbf{VIRTUAL memory} for now:
\begin{itemize}
	\item 3M, code for kernel before MMU(will be clear after MMU is enabled).
	\item	2G to 2G+512M, image for RAM
	\item	2G+2M, where the rest of kernel is load.
	\item	0xE0000000 to 0xFFFFFFFF, mapping to the corresponding place, care for the device and other things.
	\item	0xDFF00000, kernel stack during initialization, mapping physical RAM 511-512M
	\item	0xFFFF0000, interrupt table.
\end{itemize}

\textbf{PHYSICAL memory} for now:
\begin{itemize}
	\item 3M, code for kernel before MMU
	\item	2M, rest of the kernel
	\item	0xE0000000 to 0xFFFFFFFF, care for the device and other things.
	\item	511-512M, first the firmware, then the stack for sys
	\item	4M, interrupt table
\end{itemize}

\subsection{STEP 4 --- exception handler}
%todo

\section{Recently on OS}

Here we will first conclude the overall project and then high light some parts. The following part is a teamwork output, with ---todo--- implement sched related thing, ---todo--- code the file system, and three of us design the system call together. 

\subsection{How the program is run}
\begin{enumerate}
	\item	Go from firmware.bin
	\item	Fall in the MBR code, it automatically load 2 parts(before and after MMU) to the RAM.
	\item Init for all kind of device again.
	\item Enable MMU.
	\item	Enable l1cache.
	\item	Enable SCU.
	\item	Enable page manage, for page alloc
	\item	Init for slab allocator.
	\item	Init for interrupt
	\item	Init for file system
	\item Init for sheduler
	\item	Fork and exec a user program.
\end{enumerate}

\subsection{System Call}
Here we support all kind of system call, \emph{some of which is not debugged}, see the file \emph{syscall.h} for detail.

\begin{enumerate}
	\item	process related, \emph{fork(), exec(), exit(), wait(), getpid(), getppid()...}
	\item I/O and file system related, \emph{gets(), puts(), open(), close(), write(), create(), delete()...}
\end{enumerate}

A system call is actually a ``SVC ID'' instruction here, with register R0 (and probably R1 R2) to pass parameter, and R0 serves as the return value.

\subsection{Scheduler}
Here we use kind of Round Robin policy, with no priority. When private timer ticks, we run scheduler to choose next process.

\subsection{File System}
%todo

\subsection{Exception}

\subsection{Device Update}

We also made some modification to the given device, like:

\begin{enumerate}
	\item	Manage the cross frame interrupt, in sd\_dma\_spin\_read(), we have discuss it in class, and the final solution is provided by ---todo---.
	\item	For easy to use and debug, we have \emph{printf()} thing implemented.
	\item a function called \emph{puthex()} failed on my OS, because the string that puts \emph{is not end with $\backslash 0$}.
	\item \emph{gets()}, provided by ---todo---.
\end{enumerate}

\subsection{Division on ARM}

ARM architecture alone did not provide integer division, here I implement my division.\cite{---todo---}

You can click that link and view the source code. The function I implemented are \emph{\_\_aeabi\_idiv()} or \emph{\_\_aeabi\_uidiv()}.

User in our OS can simply use division, and the compiler will care about the rest. For \underline{divided by zero}, we will return 0. You are welcome the raise error for this.

\section{Others}

Necessary \emph{user guides and manual} during the development.

\begin{enumerate}
	\item	ARM® Architecture Reference Manual(ARMv7-A and ARMv7-R edition), \emph{manual for the architecture}.
	\item CortexTM-A9 MPCore® Revision: r3p0 Technical Reference Manual, \emph{manual for the specific mpcore things, such as timer, snoop control unit...}
	\item	Zynq-7000 All Programmable SoC Technical Reference Manual, \emph{about the board.}
	\item	Executable and Linkable Format.PDF, \emph{to help the load elf thing.}
	\item ARM Generic Interrupt Controller v1.0, \emph{may be good for the interrupt handling.}
\end{enumerate}
%question here

\section{Summary}
%experience
%hard
%take time
%interesting

\section{Thank You}

\indent \indent Report generated in \LaTeX

View this project on GitHub.

\begin{thebibliography}{99}
\end{thebibliography}

\end{document}